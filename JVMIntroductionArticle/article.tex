\documentclass[]{scrartcl}

% Поддержка русского языка
\usepackage[utf8]{inputenc}
\usepackage[russian]{babel}
\usepackage{setspace,amsmath}

%opening
\title{Компиляция и исполнение Java приложений}
\author{Голов Павел}

% Настройки полей документа
\usepackage{indentfirst}

% Графические пакеты
\usepackage[dvips]{graphicx}
\graphicspath{{img/}}

% Пакет для отображения кода
\usepackage{listings}

\begin{document}

\maketitle

\begin{center}
	Статья на конкурс для сайта JavaRush.
\end{center}

\thispagestyle{empty}

\newpage

\section{Вступление}

Всем привет! Сегодня я хотел бы поделиться знаниями о том, что происходит под капотом JVM (Java Virtual Machine) после того, как мы запускаем написанное Java приложение. В наше время существуют моднейшие среды разработки, которые помогают не думать о внутреннем устройстве JVM, компиляции и выполнении Java - кода, из за чего новые разработчики могут упустить эти важные аспекты. В то же время, на собеседованиях часто задают вопросы касательно этой темы,  из-за чего я и решил написать статью.

\section{Основная часть}

\subsection{Компиляция в байт-код}

Начнем с теории. Когда мы пишем какое-либо приложение, мы создаем файл с расширением .java и помещаем в него код на языке программирования Java. Такой файл, содержащий код, понятный человеку, называется файлом с исходным кодом. После того, как файл с исходным кодом готов, нужно его выполнить! Но на данной стадии в нем содержится информация, понятная только человеку. 

Java - мультиплатформенный язык программирования. Это значит, что программы, написанные на языке Java, можно выполнять на любой платформе, где установлена специальная исполняющая система Java. Такая система называется Java Virtual Machine (JVM). Для того, чтобы перевести программу из исходного кода в код, понятный JVM, нужно её скомпилировать. Код, понятный JVM называется байт-кодом и содержит набор инструкций, которые в дальнейшем будет исполнять виртуальная машина.

Для компиляции исходного кода в байт-код существует компилятор javac, входящий в поставку JDK (Java Development Kit). На вход компилятор принимает файл с расширением .java, содежащий исходный код программы, а на выходе выдает файл с расширением .class, содержащий байт-код, необходимый для исполнения программы виртуальной машиной.

После того, как программа была скомпилирована в байт код, она может быть выполнена с помощью виртуальной машины.

\subsection{Пример компиляции}

Предположим, что у нас есть простая программа, содержащаяся в файле Calculator.java, которая принимает 2 численных аргумента командной строки и печатает результат их сложения:

\begin{lstlisting}
class Calculator {
	public static void main(String[] args){
		int a = Integer.valueOf(args[0]);
		int b = Integer.valueOf(args[1]);

		System.out.println(a + b);
	}
}
\end{lstlisting}

Для того, чтобы скомпилировать эту программу в байт код, воспользуемся компилятором javac в командной строке:

\begin{lstlisting}
javac Calculator.java
\end{lstlisting}

После компиляции на выходе мы получаем файл с байт-кодом Calculator.class, который мы можем исполнить при помощи установленной на нашем компьютере java-машины при помощи команды java в командной строке:

\begin{lstlisting}
java Calculator 1 2
\end{lstlisting}

Заметим, что после названия класса были указаны 2 аргумента командной строки - числа 1 и 2. После выполнения программы, в командной строке выведется число 3.

\subsection{Исполнение программы виртуальной машиной}

\subsection{Just-in-time (JIT) компиляция}

\section{Заключение}

\end{document}
